\documentclass[dvipdfmx,12pt]{beamer}

\usepackage{bxdpx-beamer}% dvipdfmxなので必要
\usepackage{pxjahyper}% 日本語で'しおり'したい
\usepackage{minijs}% min10ヤダ
\renewcommand{\kanjifamilydefault}{\gtdefault}% 既定をゴシック体に

\usetheme[option]{PaloAlto}

\usepackage{amsmath}
\usepackage{amssymb}
\usepackage{amsfonts}
%\usepackage{euler}
\usepackage{graphicx}
\usepackage{color}
\usepackage[cache=true]{minted}

\logo{{\large$\mathbb{N}$}}
\title[ヒューマンセンシング]{主専攻実験 最終報告会 \\ ヒューマンセンシング}
\author[荻野夏樹]{情報科学類 201611353 \\ 荻野夏樹}
\date{\today}
\begin{document}
\maketitle
\section{section}
\subsection{subsection}
\begin{frame}
	\frametitle{frame1}
唐突な数式
% \hyperlink{リンク先のラベル名}{\beamergotobutton{表示するテキスト}}
\begin{eqnarray*}
	log \prod D(x)(1-D(G(z))) &=& \sum log D(x)(1-D(G(z))) \\
	                          &=& \sum log D(x) + log (1-D(G(z)))
\end{eqnarray*}
\pause
\begin{eqnarray*}
	\pi
\end{eqnarray*}
\end{frame}
%\section{箇条書き}
%\begin{frame}
%\frametitle{箇条書き}
%\begin{enumerate}
%\item ああああ
%\item いいいいい
%\pause
%\item ううううう
%\pause
%\item えええええ
%\pause
%\end{enumerate}
%\end{frame}
%\newpage
%\begin{figure}[htbp]
%\begin{center}
%\includegraphics[width=7.0cm]{./di0.png}
%\end{center}
%\end{figure}
\end{document}
